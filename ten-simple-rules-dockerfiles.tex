% Template for PLoS
% Version 3.5 March 2018
%
% % % % % % % % % % % % % % % % % % % % % %
%
% -- IMPORTANT NOTE
%
% This template contains comments intended
% to minimize problems and delays during our production
% process. Please follow the template instructions
% whenever possible.
%
% % % % % % % % % % % % % % % % % % % % % % %
%
% Once your paper is accepted for publication,
% PLEASE REMOVE ALL TRACKED CHANGES in this file
% and leave only the final text of your manuscript.
% PLOS recommends the use of latexdiff to track changes during review, as this will help to maintain a clean tex file.
% Visit https://www.ctan.org/pkg/latexdiff?lang=en for info or contact us at latex@plos.org.
%
%
% There are no restrictions on package use within the LaTeX files except that
% no packages listed in the template may be deleted.
%
% Please do not include colors or graphics in the text.
%
% The manuscript LaTeX source should be contained within a single file (do not use \input, \externaldocument, or similar commands).
%
% % % % % % % % % % % % % % % % % % % % % % %
%
% -- FIGURES AND TABLES
%
% Please include tables/figure captions directly after the paragraph where they are first cited in the text.
%
% DO NOT INCLUDE GRAPHICS IN YOUR MANUSCRIPT
% - Figures should be uploaded separately from your manuscript file.
% - Figures generated using LaTeX should be extracted and removed from the PDF before submission.
% - Figures containing multiple panels/subfigures must be combined into one image file before submission.
% For figure citations, please use "Fig" instead of "Figure".
% See http://journals.plos.org/plosone/s/figures for PLOS figure guidelines.
%
% Tables should be cell-based and may not contain:
% - spacing/line breaks within cells to alter layout or alignment
% - do not nest tabular environments (no tabular environments within tabular environments)
% - no graphics or colored text (cell background color/shading OK)
% See http://journals.plos.org/plosone/s/tables for table guidelines.
%
% For tables that exceed the width of the text column, use the adjustwidth environment as illustrated in the example table in text below.
%
% % % % % % % % % % % % % % % % % % % % % % % %
%
% -- EQUATIONS, MATH SYMBOLS, SUBSCRIPTS, AND SUPERSCRIPTS
%
% IMPORTANT
% Below are a few tips to help format your equations and other special characters according to our specifications. For more tips to help reduce the possibility of formatting errors during conversion, please see our LaTeX guidelines at http://journals.plos.org/plosone/s/latex
%
% For inline equations, please be sure to include all portions of an equation in the math environment.
%
% Do not include text that is not math in the math environment.
%
% Please add line breaks to long display equations when possible in order to fit size of the column.
%
% For inline equations, please do not include punctuation (commas, etc) within the math environment unless this is part of the equation.
%
% When adding superscript or subscripts outside of brackets/braces, please group using {}.
%
% Do not use \cal for caligraphic font.  Instead, use \mathcal{}
%
% % % % % % % % % % % % % % % % % % % % % % % %
%
% Please contact latex@plos.org with any questions.
%
% % % % % % % % % % % % % % % % % % % % % % % %

\documentclass[10pt,letterpaper]{article}
\usepackage[top=0.85in,left=2.75in,footskip=0.75in]{geometry}

% amsmath and amssymb packages, useful for mathematical formulas and symbols
\usepackage{amsmath,amssymb}

% Use adjustwidth environment to exceed column width (see example table in text)
\usepackage{changepage}

% Use Unicode characters when possible
\usepackage[utf8x]{inputenc}

% textcomp package and marvosym package for additional characters
\usepackage{textcomp,marvosym}

% cite package, to clean up citations in the main text. Do not remove.
% \usepackage{cite}

% Use nameref to cite supporting information files (see Supporting Information section for more info)
\usepackage{nameref,hyperref}

% line numbers
\usepackage[right]{lineno}

% ligatures disabled
\usepackage{microtype}
\DisableLigatures[f]{encoding = *, family = * }

% color can be used to apply background shading to table cells only
\usepackage[table]{xcolor}

% array package and thick rules for tables
\usepackage{array}

% create "+" rule type for thick vertical lines
\newcolumntype{+}{!{\vrule width 2pt}}

% create \thickcline for thick horizontal lines of variable length
\newlength\savedwidth
\newcommand\thickcline[1]{%
  \noalign{\global\savedwidth\arrayrulewidth\global\arrayrulewidth 2pt}%
  \cline{#1}%
  \noalign{\vskip\arrayrulewidth}%
  \noalign{\global\arrayrulewidth\savedwidth}%
}

% \thickhline command for thick horizontal lines that span the table
\newcommand\thickhline{\noalign{\global\savedwidth\arrayrulewidth\global\arrayrulewidth 2pt}%
\hline
\noalign{\global\arrayrulewidth\savedwidth}}


% Remove comment for double spacing
%\usepackage{setspace}
%\doublespacing

% Text layout
\raggedright
\setlength{\parindent}{0.5cm}
\textwidth 5.25in
\textheight 8.75in

% Bold the 'Figure #' in the caption and separate it from the title/caption with a period
% Captions will be left justified
\usepackage[aboveskip=1pt,labelfont=bf,labelsep=period,justification=raggedright,singlelinecheck=off]{caption}
\renewcommand{\figurename}{Fig}

% Use the PLoS provided BiBTeX style
% \bibliographystyle{plos2015}

% Remove brackets from numbering in List of References
\makeatletter
\renewcommand{\@biblabel}[1]{\quad#1.}
\makeatother



% Header and Footer with logo
\usepackage{lastpage,fancyhdr,graphicx}
\usepackage{epstopdf}
%\pagestyle{myheadings}
\pagestyle{fancy}
\fancyhf{}
%\setlength{\headheight}{27.023pt}
%\lhead{\includegraphics[width=2.0in]{PLOS-submission.eps}}
\rfoot{\thepage/\pageref{LastPage}}
\renewcommand{\headrulewidth}{0pt}
\renewcommand{\footrule}{\hrule height 2pt \vspace{2mm}}
\fancyheadoffset[L]{2.25in}
\fancyfootoffset[L]{2.25in}
\lfoot{\today}

%% Include all macros below

\newcommand{\lorem}{{\bf LOREM}}
\newcommand{\ipsum}{{\bf IPSUM}}





\usepackage{forarray}
\usepackage{xstring}
\newcommand{\getIndex}[2]{
  \ForEach{,}{\IfEq{#1}{\thislevelitem}{\number\thislevelcount\ExitForEach}{}}{#2}
}

\setcounter{secnumdepth}{0}

\newcommand{\getAff}[1]{
  \getIndex{#1}{}
}

\providecommand{\tightlist}{%
  \setlength{\itemsep}{0pt}\setlength{\parskip}{0pt}}

\begin{document}
\vspace*{0.2in}

% Title must be 250 characters or less.
\begin{flushleft}
{\Large
\textbf\newline{Ten Simple Rules for Writing Dockerfiles for Reproducible Research} % Please use "sentence case" for title and headings (capitalize only the first word in a title (or heading), the first word in a subtitle (or subheading), and any proper nouns).
}
\newline
% Insert author names, affiliations and corresponding author email (do not include titles, positions, or degrees).
\\
Daniel Nüst\textsuperscript{\getAff{Institute for Geoinformatics, University of M``unster, M''unster,
Germany}}\textsuperscript{*},
Stephen Eglen\textsuperscript{\getAff{Another University}}\\
\bigskip
\bigskip
* Corresponding author: daniel.nuest@uni-muenster.de\\
\end{flushleft}
% Please keep the abstract below 300 words
\section*{Abstract}
Containerisation is a useful concept for capturing the increasingly
complex virtual laboratories that underpin computational sciences today.
Docker is the most widely used containerisation solution. A Docker image
is created based on the instructions in a plain-text file in the
\texttt{Dockerfile} format. In a scholarly context, transparency and
support of reproducibility are most desired aspects of containers. By
following the rules in this article researchers writing a
\texttt{Dockerfile} can increase the changes to apply containers
effectily in their daily work but also enable fellow researchers to
reproduce a workflow.

% Please keep the Author Summary between 150 and 200 words
% Use first person. PLOS ONE authors please skip this step.
% Author Summary not valid for PLOS ONE submissions.
\section*{Author summary}
TBD

\linenumbers

% Use "Eq" instead of "Equation" for equation citations.
\hypertarget{introduction}{%
\section*{Introduction}\label{introduction}}
\addcontentsline{toc}{section}{Introduction}

\begin{itemize}
\tightlist
\item
  containerisation is a useful and powerful to handle increasingly
  complex virtual laboratories for computational sciences
\item
  to some extent all sciences today use algorithms to analyse data
\item
  researchers need skills to handle virtual laboratories
\item
  well-defined computational environments are crucial for
  reproducibility
\item
  portable computational environments are crucial for transparency,
  e.g.~in peer-review
\item
  containerisation can help to work towards the ideal of transparence an
  reproducibility
\item
  Docker is a common containerisation solution, widely adopted in
  mainstream IT and therefore widely available and support on many
  platforms, which makes it usable for non-IT experts in science
\item
  Dockerfiles are machine- and human-readable recipes for creating a
  container
\item
  In RR, Dockerfiles can document where data and code came from and
  likely also where a third party might still get them
\item
  this article takes a look at how to write a \texttt{Dockerfile} so
  that it facilitates a day-to-day research workflow as well as the
  higher goals of reproducibility
\item
  while you can interactively manipulate a container, you never should
  (\texttt{docker\ commit})
\item
  \emph{conventions} lead to readability by others (potentially reuse)
\end{itemize}

\begin{center}\rule{0.5\linewidth}{\linethickness}\end{center}

\textbf{Box 1: Non-Docker container containerisation tools}

\begin{itemize}
\tightlist
\item
  Singularity can import Docker containers, though rules transferable to
  ``the singularity recipee''
\end{itemize}

\begin{center}\rule{0.5\linewidth}{\linethickness}\end{center}

\hypertarget{use-versioned-and-automatically-built-base-images}{%
\section*{1. Use versioned and automatically built base
images}\label{use-versioned-and-automatically-built-base-images}}
\addcontentsline{toc}{section}{1. Use versioned and automatically built
base images}

\begin{itemize}
\tightlist
\item
  understand how base images work
\item
  \emph{never} use \texttt{:latest}
\item
  use Linux distribution that supports the required software stack, and
  ideally that is widely used in your community
\item
  base images (all the way to the top) must be based on Dockerfiles
  themselves
\item
  library base images are well maintained and security tested, but
  alternatives might be more suitable for research purposes / RR
  (example \texttt{rocker/r-ver})
\item
  base images that have complex software installed (e.g.~ML libraries,
  specific BLAS library) are helpful and fine to use, just ensure there
  is a Dockerfile publicly available that they use (and add a link to
  that file in your Dockerfile)
\item
  ideally the images are maintained by an active community/your
  community
\end{itemize}

\hypertarget{use-indentation-newlines-and-comments-for-documentation-readability-and-structure}{%
\section{2. Use indentation, newlines, and comments for documentation,
readability and
structure}\label{use-indentation-newlines-and-comments-for-documentation-readability-and-structure}}

\begin{itemize}
\tightlist
\item
  can use comments to add sections to the Dockerfile to reduce the need
  to externalise when files get long
\item
  carefully indent commands and their arguments to make clear what
  belongs together, especially when connecting multiple commands in onr
  \texttt{RUN} with \texttt{\&\&}
\item
  use \texttt{\textbackslash{}} for newlines
\item
  use a linter to follow common practices and consistency
\item
  \textbf{Use comments to document decisions and usage}

  \begin{itemize}
  \tightlist
  \item
    make the \texttt{Dockerfile} self-explanatory by adding comments for
    specific decisions
  \item
    add reasons and links to followed tutorials (failed attempts may be
    found in the history)
  \item
    similar to ``literate programming''
  \item
    put \texttt{docker\ run} and \texttt{docker\ build} commands in
    comments at the end of the file (\emph{may be own rule?}),
    especially relevant if arguments are used
  \item
    examples are especially crucial if you require configuration e.g.~of
    the user
  \end{itemize}
\end{itemize}

\hypertarget{pin-versions}{%
\section*{3. Pin versions}\label{pin-versions}}
\addcontentsline{toc}{section}{3. Pin versions}

\textbf{system libraries}

\begin{itemize}
\tightlist
\item
  you can install specific versions of system packages with the
  respective package manager, also called version pinning

  \begin{itemize}
  \tightlist
  \item
    on apt: https://blog.backslasher.net/my-pinning-guidelines.html
  \end{itemize}
\item
  do so if the version is relevant, e.g.~to demonstrate a bug, or likely
  to become a problem, e.g.~because of \ldots{}
\item
  do so if you are aware of the system library being relevant to your
  workflow
\item
  you can find out about the currently installed versions

  \begin{itemize}
  \tightlist
  \item
    Debian/Ubuntu: \texttt{dpkg\ -\/-list}
  \item
    Alpine: \texttt{apk\ -vv\ info\textbar{}sort}
  \item
    CentOS: \texttt{yum\ list\ installed} or \texttt{rpm\ -qa}
  \end{itemize}
\item
  \emph{installing from source} is a useful way to install very specific
  versions, at the cost of needing build libraries (which could be
  removed again with layered builds)
\end{itemize}

\textbf{extension packages and programming language modules}

\begin{itemize}
\tightlist
\item
  package managers of programming languages are a good solution to
  install a collection of dependencies for a language
\item
  package managers have a CLI and can be used from \texttt{RUN} commands
\item
  there is a risk in outsourcing configuration to the file formats
  supported by package managers \textgreater{} use only when direct
  installation in Dockerfile becomes complex; example files:

  \begin{itemize}
  \tightlist
  \item
    Python: \texttt{requirements\ .txt}, \texttt{xxx.yml} (Conda)
  \item
    R: \texttt{DESCRIPTION}
  \item
    Java: \texttt{mvn.xml}
  \item
    JavaScript: \texttt{package.json} of \texttt{npm}
  \end{itemize}
\item
  how to do in Python (\texttt{==\ x.y.z})
\item
  do it in R with \texttt{versions} package, or by using MRAN (e.g.~via
  \texttt{r-ver} image)- JavaScript?
\item
  Julia: \texttt{add\ Package@1.0} \textgreater{}
  https://julialang.github.io/Pkg.jl/v1/managing-packages/\#Adding-packages-1
\item
  Use common command-line ready installation commands of programming
  languages

  \begin{itemize}
  \tightlist
  \item
    better readbiliy, potentially even performance
    (\texttt{RUN\ install2.r\ sp} instead of
    \texttt{RUN\ R\ -e\ "install.packages(\textquotesingle{}sf\textquotesingle{})"},
    although the latter is ``base R'')
  \end{itemize}
\end{itemize}

\hypertarget{mount-data-and-control-code}{%
\section*{4. Mount data and control
code}\label{mount-data-and-control-code}}
\addcontentsline{toc}{section}{4. Mount data and control code}

\begin{itemize}
\tightlist
\item
  do not use \texttt{ADD}/\texttt{COPY} to insert data or code into an
  image
\item
  better mount them to have them outside of the image
\item
  easier access, does not require Docker knowledge by third parties to
  reause code and data
\item
  be always ready to throw containers and images away
\item
  use the \texttt{-\/-user} option to avoid problems with file
  permissions when mounting
\item
  if you have a ``stable'' published software library, install it from
  source from the source code repo or from the software repository (so
  that users find the project in the future)
\end{itemize}

\hypertarget{only-switch-directoryies-with-workdir}{%
\section*{\texorpdfstring{5. Only switch directoryies with
\texttt{WORKDIR}}{5. Only switch directoryies with WORKDIR}}\label{only-switch-directoryies-with-workdir}}
\addcontentsline{toc}{section}{5. Only switch directoryies with
\texttt{WORKDIR}}

\begin{itemize}
\tightlist
\item
  might need to move to different directories for bespoke configuration
  or building from source
\item
  is is much more transparent than \texttt{cd\ X} or \texttt{cd\ ...} in
  \texttt{RUN} statements
\item
  clarity most important
\end{itemize}

\hypertarget{use-labels-and-build-arguments-for-relevant-links-and-metadata}{%
\section*{6. Use labels and build arguments for relevant links and
metadata}\label{use-labels-and-build-arguments-for-relevant-links-and-metadata}}
\addcontentsline{toc}{section}{6. Use labels and build arguments for
relevant links and metadata}

\begin{itemize}
\tightlist
\item
  advantage of labels: are structured, can be exposed by APIs,
  e.g.~https://microbadger.com/labels
\item
  use namespaced-names
\item
  http://label-schema.org/rc1/ respectively
  https://github.com/opencontainers/image-spec
\item
  repository link for Dockerfile
\item
  author (\texttt{MAINTAINER} is deprecated)
\item
  license
\item
  usage instructions
\item
  https://microbadger.com/labels
\item
  DOI of research compendium (Zenodo preregister instead of GitHub
  automatic integration)
\item
  \textbf{Use build arguments to capture build metadata}

  \begin{itemize}
  \tightlist
  \item
    add git commit hash to label
  \item
    add date and time to label
  \end{itemize}
\end{itemize}

\hypertarget{enable-both-interactive-development-and-one-click-execution}{%
\section*{7. Enable both interactive development and one-click
execution}\label{enable-both-interactive-development-and-one-click-execution}}
\addcontentsline{toc}{section}{7. Enable both interactive development
and one-click execution}

\begin{itemize}
\tightlist
\item
  using \texttt{CMD} and \texttt{ENTRYPOINT} make sure that it is
  possible to run the container interactively \textgreater{} give
  examples (see below)
\item
  the default execution should either execute the workflow (headless) or
  start an analysis environment

  \begin{itemize}
  \tightlist
  \item
    if your workflow/sofware does not support headless execution
    (Excel?), switch tools
  \item
    or have default with UI and only document headless execution via
    example commands
  \end{itemize}
\item
  may also use the same \texttt{Dockerfile} for different purposes,
  e.g.~include an app (e.g.~Shiny) for interactive use by user
\item
  document both variants with example commands
\item
  a headless execution can be executed in a continuous integration after
  every project update, potentially on a test dataset for speed-up
\end{itemize}

\hypertarget{favour-clarity-over-image-size}{%
\section*{8. Favour clarity over image
size}\label{favour-clarity-over-image-size}}
\addcontentsline{toc}{section}{8. Favour clarity over image size}

\begin{itemize}
\tightlist
\item
  don't worry about image size (i.e.~no complex RUN commands that remove
  files rightaway) but order instructions
\item
  clarity more important
\item
  how to access the layer commands from an existing image
  (\texttt{docker\ inspect..})
\item
  have commands \emph{in order} of least likely to change to most likely
  to change \textgreater{} even helps readers!

  \begin{enumerate}
  \def\labelenumi{\arabic{enumi}.}
  \tightlist
  \item
    system libraries
  \item
    language-specific libraries or modules
  \item
    from repositories (binaries)
  \item
    from source
  \item
    own software/scripts (if not mounted)
  \item
    labels
  \item
    \texttt{RUN}/\texttt{ENTRYPOINT}
  \end{enumerate}
\item
  if need be use \emph{layered builds} to only keep specific files from
  one build step to another
\end{itemize}

\hypertarget{publish-a-dockerfile-per-project-in-a-code-repository-with-version-control}{%
\section*{9. Publish a Dockerfile per project in a code repository with
version
control}\label{publish-a-dockerfile-per-project-in-a-code-repository-with-version-control}}
\addcontentsline{toc}{section}{9. Publish a Dockerfile per project in a
code repository with version control}

\begin{itemize}
\tightlist
\item
  \texttt{Dockerfile} is a plain text-based format and therefore you
  should put it under version control
\item
  add the link to the online repository to a label, to point back to the
  source of the file
\item
  versioning on a collaboration platform exposes your environment
  configuration and enables collaboration/feedback
\item
  you can build and run (e.g.~on a test dataset!) you Dockerfile in CI
  (cf.~automation below)
\item
  keep \texttt{Dockerfile} in the same project with your workflow and
  data (cf.~research compendium concept?)
\item
  \textbf{this should be the repository with the workflow and data}
  (cf.~research compendium)

  \begin{itemize}
  \tightlist
  \item
    Use one \texttt{Dockerfile} per workflow or project and put one
    ``thing'' in; \textbf{TO DISCUSS}: argue against the above rule and
    recommend having a process manager and multiple processes in one
    container
  \item
    start with a clean slate for a new project - shared lines are
    quickly copied over, and Docker's build caching will bring some
    performance
  \item
    allows you to quickly switch between projects and not worry about
    breaking things you are not working on
  \item
    have one obvious main process per project, e.g. \texttt{R} or
    RStudio
  \item
    if you have a complex set-up of several proecceses, e.g.~with a
    database, then put it in a separate container and connect them via
    \texttt{docker-compose}
  \end{itemize}
\end{itemize}

\hypertarget{use-the-container-daily-rebuild-the-image-weekly}{%
\section*{10. Use the container daily, rebuild the image
weekly}\label{use-the-container-daily-rebuild-the-image-weekly}}
\addcontentsline{toc}{section}{10. Use the container daily, rebuild the
image weekly}

\begin{itemize}
\tightlist
\item
  use the container built by the \texttt{Dockerfile} in your regular
  work, it is the only way to make sure it is really stable
  (cf.~Marwick's ``this container is the only way I have ever run this
  workflow'')
\item
  no showstopper for using UIs (web-based, e.g.~Jupyter, RStudio, but
  also \texttt{x11docker})
\item
  during development and analysis, interactive use (e.g.~R session,
  Jupyter Notebook) has advantages, and even the most disciplined might
  install a package or change a parameter manually
\item
  regularly delete all containers and rebuild images based on your
  \texttt{Dockerfile}
\item
  you are more likely to remember the undocumented steps if done
  regularly
\item
  increases trust in configuration, encourages effetiveness and fully
  scripted configuration
\item
  keep a \texttt{Makefile} next to the Dockerfile so you don't fall into
  the trap of not regularly rebuilding your digital laboratory (better
  to have build and run commands - i.e.~the usage - in two places and
  potentially diverging than the actual \texttt{Dockerfile})
\item
  \textbf{Don't replicate environment configuration outside of the
  Dockerfile for convenience}

  \begin{itemize}
  \tightlist
  \item
    make the Dockerfile work for your day-to-day research instead of
    having a second set of configurations in on the ``local'' machine
  \item
    having two approaches will eventually break, only a perceived
    convenience
  \item
    avoid an untidy laboratory in practice behind a shiny appearance of
    a \texttt{Dockerfile}
  \item
    you can install interactive UIs as part of the Dockerfile and use
    them just like Desktop UIs (Jupyter, RStudio, use )
  \end{itemize}
\end{itemize}

\begin{center}\rule{0.5\linewidth}{\linethickness}\end{center}

\textbf{Box 2: Automatic generation of Dockerfiles}

\begin{itemize}
\tightlist
\item
  there are tools you can auto-generate a \texttt{Dockerfile}
\item
  can be a good as a starting point, careful to avoid a lock-in
\item
  they have limitations, namely \ldots{}
\item
  \texttt{repo2docker}, \texttt{dockter}, \texttt{containerit}
\item
  these are useful if you don't need very specific versions etc. and for
  specific use cases, but sometimes requires a specific project
  structure (PyPI \texttt{requirements.txt}) or reproducible document (R
  Markdown file)
\end{itemize}

\begin{center}\rule{0.5\linewidth}{\linethickness}\end{center}

\hypertarget{conclusion}{%
\section*{Conclusion}\label{conclusion}}
\addcontentsline{toc}{section}{Conclusion}

\begin{itemize}
\tightlist
\item
  reproducibility is about best efforts, not about achieving the perfect

  \begin{itemize}
  \tightlist
  \item
    https://twitter.com/DougBlank/status/1135904909663068165?s=09
  \end{itemize}
\item
  don't go insane, but be realistic about what might break and what is
  unlikely to break
\item
  all the rules can be broken if another way works better for \emph{you}
\item
  document for your future self, provide detailed docs only if others
  ask for it {[}REF{]}
\end{itemize}

\hypertarget{acknowledgements}{%
\section*{Acknowledgements}\label{acknowledgements}}
\addcontentsline{toc}{section}{Acknowledgements}

\hypertarget{references}{%
\section*{References}\label{references}}
\addcontentsline{toc}{section}{References}

\nolinenumbers


\end{document}

